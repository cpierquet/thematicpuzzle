% !TeX TXS-program:compile = txs:///arara
% arara: pdflatex: {shell: no, synctex: no, interaction: batchmode}
% arara: pdflatex: {shell: no, synctex: no, interaction: batchmode} if found('log', '(undefined references|Please rerun|Rerun to get)')

\documentclass[11pt,a4paper]{ltxdoc}
\usepackage{thematicpuzzle}
\usepackage{fancyvrb}
\usepackage{fancyhdr}
\usepackage{hyperref}
\usepackage{tcolorbox}
\tcbuselibrary{listings,listingsutf8}
\fancyhf{}
\renewcommand{\headrulewidth}{0pt}
\lfoot{\sffamily\small [thematicpuzzle]}
\rfoot{\sffamily\small - \thepage{} -}
\usepackage{hologo}
\providecommand\tikzlogo{Ti\textit{k}Z}
\providecommand\TeXLive{\TeX{}Live\xspace}
\providecommand\PSTricks{\textsf{PSTricks}\xspace}
\let\pstricks\PSTricks
\let\TikZ\tikzlogo

\urlstyle{same}
\hypersetup{pdfborder=0 0 0}
\usepackage[margin=2cm]{geometry}
\setlength{\parindent}{0pt}
\def\TPversion{0.1.0}
\def\TPdate{17/12/2023}
\usepackage{soul}
\usepackage{codehigh}
\usepackage{tabularray}
\sethlcolor{lightgray!25}
\NewDocumentCommand\MontreCode{ m }{%
	\hl{\vphantom{\texttt{pf}}\texttt{#1}}%
}
\usepackage[french]{babel}

\begin{document}

\pagestyle{fancy}

\thispagestyle{empty}

\begin{center}
	\begin{minipage}{0.88\linewidth}
	\begin{tcolorbox}[colframe=yellow,colback=yellow!15]
		\begin{center}
			\begin{tabular}{c}
				{\Huge \texttt{thematicpuzzle [fr]}}\\
				\\
				{\LARGE Présenter un 'puzzle' horizontal,} \\
				{\LARGE avec différents thèmes.} \\
				\\
				{\small \texttt{Version \TPversion{} -- \TPdate}}
		\end{tabular}
		\end{center}
	\end{tcolorbox}
\end{minipage}
\end{center}

\begin{center}
	\begin{tabular}{c}
	\texttt{Cédric Pierquet}\\
	{\ttfamily c pierquet -- at -- outlook . fr}\\
	\texttt{\url{https://github.com/cpierquet/thematicpuzzle}} \\
\end{tabular}
\end{center}

\hrule

\vfill

\begin{tcolorbox}[colframe=lightgray,colback=lightgray!5,halign=center]
\PuzzleThemes{A,B,C,D,E,F,G,H}
\end{tcolorbox}

\begin{tcolorbox}[colframe=lightgray,colback=lightgray!5,halign=center]
\PuzzleThemes[Epaisseur=1.25pt,Echelle=2.75,CouleursFond={yellow!10,orange!10,green!10,teal!10,purple!10},Labels={Probas,Stats,Python,Suites,Matrices},CouleurBord=darkgray,CouleurIcones=teal!50!black]{\ThemProbas,\ThemStats,\ThemPython,\ThemSuites,\ThemMatrices}
\end{tcolorbox}

\vfill~

\pagebreak

\phantomsection

\hypertarget{matoc}{}

\tableofcontents

\vspace*{5mm}

\hrule

\vspace*{5mm}

\section{Introduction}

\subsection{Description}

Avec ce package, il est possible de créer un \textit{bandeau horizontal}, sous forme de puzzle, avec différentes thématiques.

Les styles sont \textit{globalement} fixés, mais il est possible de configurer :

\begin{itemize}
	\item les couleurs ;
	\item les icones ;
	\item les labels.
\end{itemize}

Les \textit{bordures} intérieures sont aléatoirement orientées vers la gauche ou vers la droite, et le code se charge de colorier (le cas échéant) les pièces de manière satisfaisante.

\subsection{Décorations}

Pour le moment, les \textit{décorations} des pièces sont à déclarer comme des caractères, par exemple avec des icônes du package \MontreCode{fontawesome5} (chargé par le package).

\smallskip

À terme, il sera peut-être possible d'insérer des images ou des codes \TikZ, cela fait partie des pistes d'améliorations.

\subsection{Chargement}

Pour charger le package, il suffit d'ajouter dans le préambule :

\begin{quote}
\begin{verbatim}
\usepackage{thematicpuzzle}
\end{verbatim}
\end{quote}

Les packages utiles, et chargés sont :

\begin{itemize}
	\item \MontreCode{tikz} ;
	\item \MontreCode{fontawesome5} ;
	\item \MontreCode{simplekv}, \MontreCode{listofitems} et \MontreCode{xstring}.
\end{itemize}

\vfill

\subsection{Historique}

\begin{quote}
\begin{verbatim}
0.1.0 : Version initiale
\end{verbatim}
\end{quote}

\pagebreak

\section{La commande}

\subsection{Fonctionnement global}

Dans le but de créer un \textit{puzzle de thèmes}, la commande est :

\begin{quote}
\begin{verbatim}
%----puzzle de thèmes
\PuzzleThemes[clés]<options tikz>{liste des décorations}
\end{verbatim}
\end{quote}

\subsection{Clés, arguments}

Les \textsf{clés}, optionnelles et entre \MontreCode{[...]}, sont :

\begin{itemize}
	\item \MontreCode{Epaisseur} := épaisseur des traits (\MontreCode{1pt} par défaut) ;
	\item \MontreCode{Echelle} := échelle globale de la figure (\MontreCode{1} par défaut) ;
	\item \MontreCode{CouleursFond} := couleur de fond (\MontreCode{vide} par défaut) ;
	\item \MontreCode{CouleurBord} := couleur(s) (si besoin séparées par des \MontreCode{,}) des bordures (\MontreCode{black} par défaut) ;
	\item \MontreCode{Labels} := libellés (optionnels) des thèmes (\MontreCode{vide} par défaut) ;
	\item \MontreCode{EchelleLabels} := échelle des libellés (\MontreCode{1.15} par défaut) ;
	\item \MontreCode{PoliceLabels} := police des libellés (\MontreCode{\textbackslash tiny\textbackslash sffamily} par défaut) ;
	\item \MontreCode{CouleurIcones} := couleur des icônes et les libellés (\MontreCode{black} par défaut).
\end{itemize}

\medskip

L'argument optionnel, entre \MontreCode{<...>}, correspond à des options spécifiques (en langage \TikZ) à passer à l'environnement créé.

\smallskip

L'argument obligatoire, entre \MontreCode{\{...\}}, correspond à la liste des décorations (sous forme de \textit{caractères}) des différentes pièces de puzzle, sous la forme \MontreCode{\{Déco<1>,Déco<2>,...,Déco<n>\}}

\subsection{Raccourcis pour certaines décorations}

Des \textit{raccourcis} francisés ont été créés dans le package, pour des saisies simplifiées d décoration :

\begin{center}
	\begin{tblr}{hlines,colspec={Q[m,l]Q[m,l]}}
		\fakeverb{\ThemProbas} & \ThemProbas \\
		\fakeverb{\ThemStats} & \ThemStats \\
		\fakeverb{\ThemChimie} & \ThemChimie \\
		\fakeverb{\ThemArithm} & \ThemArithm \\
		\fakeverb{\ThemGraphes} & \ThemGraphes \\
		\fakeverb{\ThemComplexes} & \ThemComplexes \\
		\fakeverb{\ThemMPM} & \ThemMPM \\
		\fakeverb{\ThemCalculs} & \ThemCalculs \\
		\fakeverb{\ThemMatrices} & \ThemMatrices \\
		\fakeverb{\ThemGeometrie} & \ThemGeometrie \\
		\fakeverb{\ThemEspace} & \ThemEspace \\
		\fakeverb{\ThemPhysique} & \ThemPhysique \\
		\fakeverb{\ThemFonctions} & \ThemFonctions \\
		\fakeverb{\ThemSuites} & \ThemSuites \\
		\fakeverb{\ThemTableur} & \ThemTableur \\
		\fakeverb{\ThemAlgo} & \ThemAlgo \\
		\fakeverb{\ThemEco} & \ThemEco \\
		\fakeverb{\ThemPython} & \ThemPython \\
		\fakeverb{\ThemLogique} & \ThemLogique \\
	\end{tblr}
\end{center}

\subsection{Exemples}

\begin{tcblisting}{colframe=lightgray,colback=lightgray!5}
\PuzzleThemes{\faAngry,\faArchway,\faAtlas,\faAtom}
\end{tcblisting}

\begin{tcblisting}{colframe=lightgray,colback=lightgray!5}
\PuzzleThemes[Labels={Maths,Histoire,Géo,Chimie}]{\faAngry,\faArchway,\faAtlas,\faAtom}
\end{tcblisting}

\begin{tcblisting}{colframe=lightgray,colback=lightgray!5}
\PuzzleThemes%
  [Echelle=2,CouleursFond=yellow!25,Labels={Maths,Histoire,Géo,Chimie}]%
  {\faAngry,\faArchway,\faAtlas,\faAtom}
\end{tcblisting}

\begin{tcblisting}{colframe=lightgray,colback=lightgray!5}
\PuzzleThemes%
  [Echelle=3,CouleursFond={yellow!10,orange!10,red!10,teal!10},
  Labels={Maths,Histoire,Géo,Chimie}]%
  {\faAngry,\faArchway,\faAtlas,\faAtom}
\end{tcblisting}

\begin{tcblisting}{colframe=lightgray,colback=lightgray!5}
\PuzzleThemes%
  [Epaisseur=2pt,Echelle=1.25,
  CouleursFond={yellow!10,orange!10,red!10,teal!10,purple!10},
  Labels={Probas,Stats,Python,Suites,Matrices},CouleurBord=brown,
  CouleurIcones=teal!50!black]%
  {\ThemProbas,\ThemStats,\ThemPython,\ThemSuites,\ThemMatrices}
\end{tcblisting}

\end{document}